% Options for packages loaded elsewhere
% Options for packages loaded elsewhere
\PassOptionsToPackage{unicode}{hyperref}
\PassOptionsToPackage{hyphens}{url}
\PassOptionsToPackage{dvipsnames,svgnames,x11names}{xcolor}
%
\documentclass[
  letterpaper,
  DIV=11,
  numbers=noendperiod]{scrartcl}
\usepackage{xcolor}
\usepackage{amsmath,amssymb}
\setcounter{secnumdepth}{-\maxdimen} % remove section numbering
\usepackage{iftex}
\ifPDFTeX
  \usepackage[T1]{fontenc}
  \usepackage[utf8]{inputenc}
  \usepackage{textcomp} % provide euro and other symbols
\else % if luatex or xetex
  \usepackage{unicode-math} % this also loads fontspec
  \defaultfontfeatures{Scale=MatchLowercase}
  \defaultfontfeatures[\rmfamily]{Ligatures=TeX,Scale=1}
\fi
\usepackage{lmodern}
\ifPDFTeX\else
  % xetex/luatex font selection
\fi
% Use upquote if available, for straight quotes in verbatim environments
\IfFileExists{upquote.sty}{\usepackage{upquote}}{}
\IfFileExists{microtype.sty}{% use microtype if available
  \usepackage[]{microtype}
  \UseMicrotypeSet[protrusion]{basicmath} % disable protrusion for tt fonts
}{}
\makeatletter
\@ifundefined{KOMAClassName}{% if non-KOMA class
  \IfFileExists{parskip.sty}{%
    \usepackage{parskip}
  }{% else
    \setlength{\parindent}{0pt}
    \setlength{\parskip}{6pt plus 2pt minus 1pt}}
}{% if KOMA class
  \KOMAoptions{parskip=half}}
\makeatother
% Make \paragraph and \subparagraph free-standing
\makeatletter
\ifx\paragraph\undefined\else
  \let\oldparagraph\paragraph
  \renewcommand{\paragraph}{
    \@ifstar
      \xxxParagraphStar
      \xxxParagraphNoStar
  }
  \newcommand{\xxxParagraphStar}[1]{\oldparagraph*{#1}\mbox{}}
  \newcommand{\xxxParagraphNoStar}[1]{\oldparagraph{#1}\mbox{}}
\fi
\ifx\subparagraph\undefined\else
  \let\oldsubparagraph\subparagraph
  \renewcommand{\subparagraph}{
    \@ifstar
      \xxxSubParagraphStar
      \xxxSubParagraphNoStar
  }
  \newcommand{\xxxSubParagraphStar}[1]{\oldsubparagraph*{#1}\mbox{}}
  \newcommand{\xxxSubParagraphNoStar}[1]{\oldsubparagraph{#1}\mbox{}}
\fi
\makeatother

\usepackage{color}
\usepackage{fancyvrb}
\newcommand{\VerbBar}{|}
\newcommand{\VERB}{\Verb[commandchars=\\\{\}]}
\DefineVerbatimEnvironment{Highlighting}{Verbatim}{commandchars=\\\{\}}
% Add ',fontsize=\small' for more characters per line
\usepackage{framed}
\definecolor{shadecolor}{RGB}{241,243,245}
\newenvironment{Shaded}{\begin{snugshade}}{\end{snugshade}}
\newcommand{\AlertTok}[1]{\textcolor[rgb]{0.68,0.00,0.00}{#1}}
\newcommand{\AnnotationTok}[1]{\textcolor[rgb]{0.37,0.37,0.37}{#1}}
\newcommand{\AttributeTok}[1]{\textcolor[rgb]{0.40,0.45,0.13}{#1}}
\newcommand{\BaseNTok}[1]{\textcolor[rgb]{0.68,0.00,0.00}{#1}}
\newcommand{\BuiltInTok}[1]{\textcolor[rgb]{0.00,0.23,0.31}{#1}}
\newcommand{\CharTok}[1]{\textcolor[rgb]{0.13,0.47,0.30}{#1}}
\newcommand{\CommentTok}[1]{\textcolor[rgb]{0.37,0.37,0.37}{#1}}
\newcommand{\CommentVarTok}[1]{\textcolor[rgb]{0.37,0.37,0.37}{\textit{#1}}}
\newcommand{\ConstantTok}[1]{\textcolor[rgb]{0.56,0.35,0.01}{#1}}
\newcommand{\ControlFlowTok}[1]{\textcolor[rgb]{0.00,0.23,0.31}{\textbf{#1}}}
\newcommand{\DataTypeTok}[1]{\textcolor[rgb]{0.68,0.00,0.00}{#1}}
\newcommand{\DecValTok}[1]{\textcolor[rgb]{0.68,0.00,0.00}{#1}}
\newcommand{\DocumentationTok}[1]{\textcolor[rgb]{0.37,0.37,0.37}{\textit{#1}}}
\newcommand{\ErrorTok}[1]{\textcolor[rgb]{0.68,0.00,0.00}{#1}}
\newcommand{\ExtensionTok}[1]{\textcolor[rgb]{0.00,0.23,0.31}{#1}}
\newcommand{\FloatTok}[1]{\textcolor[rgb]{0.68,0.00,0.00}{#1}}
\newcommand{\FunctionTok}[1]{\textcolor[rgb]{0.28,0.35,0.67}{#1}}
\newcommand{\ImportTok}[1]{\textcolor[rgb]{0.00,0.46,0.62}{#1}}
\newcommand{\InformationTok}[1]{\textcolor[rgb]{0.37,0.37,0.37}{#1}}
\newcommand{\KeywordTok}[1]{\textcolor[rgb]{0.00,0.23,0.31}{\textbf{#1}}}
\newcommand{\NormalTok}[1]{\textcolor[rgb]{0.00,0.23,0.31}{#1}}
\newcommand{\OperatorTok}[1]{\textcolor[rgb]{0.37,0.37,0.37}{#1}}
\newcommand{\OtherTok}[1]{\textcolor[rgb]{0.00,0.23,0.31}{#1}}
\newcommand{\PreprocessorTok}[1]{\textcolor[rgb]{0.68,0.00,0.00}{#1}}
\newcommand{\RegionMarkerTok}[1]{\textcolor[rgb]{0.00,0.23,0.31}{#1}}
\newcommand{\SpecialCharTok}[1]{\textcolor[rgb]{0.37,0.37,0.37}{#1}}
\newcommand{\SpecialStringTok}[1]{\textcolor[rgb]{0.13,0.47,0.30}{#1}}
\newcommand{\StringTok}[1]{\textcolor[rgb]{0.13,0.47,0.30}{#1}}
\newcommand{\VariableTok}[1]{\textcolor[rgb]{0.07,0.07,0.07}{#1}}
\newcommand{\VerbatimStringTok}[1]{\textcolor[rgb]{0.13,0.47,0.30}{#1}}
\newcommand{\WarningTok}[1]{\textcolor[rgb]{0.37,0.37,0.37}{\textit{#1}}}

\usepackage{longtable,booktabs,array}
\usepackage{calc} % for calculating minipage widths
% Correct order of tables after \paragraph or \subparagraph
\usepackage{etoolbox}
\makeatletter
\patchcmd\longtable{\par}{\if@noskipsec\mbox{}\fi\par}{}{}
\makeatother
% Allow footnotes in longtable head/foot
\IfFileExists{footnotehyper.sty}{\usepackage{footnotehyper}}{\usepackage{footnote}}
\makesavenoteenv{longtable}
\usepackage{graphicx}
\makeatletter
\newsavebox\pandoc@box
\newcommand*\pandocbounded[1]{% scales image to fit in text height/width
  \sbox\pandoc@box{#1}%
  \Gscale@div\@tempa{\textheight}{\dimexpr\ht\pandoc@box+\dp\pandoc@box\relax}%
  \Gscale@div\@tempb{\linewidth}{\wd\pandoc@box}%
  \ifdim\@tempb\p@<\@tempa\p@\let\@tempa\@tempb\fi% select the smaller of both
  \ifdim\@tempa\p@<\p@\scalebox{\@tempa}{\usebox\pandoc@box}%
  \else\usebox{\pandoc@box}%
  \fi%
}
% Set default figure placement to htbp
\def\fps@figure{htbp}
\makeatother





\setlength{\emergencystretch}{3em} % prevent overfull lines

\providecommand{\tightlist}{%
  \setlength{\itemsep}{0pt}\setlength{\parskip}{0pt}}



 


\KOMAoption{captions}{tableheading}
\makeatletter
\@ifpackageloaded{caption}{}{\usepackage{caption}}
\AtBeginDocument{%
\ifdefined\contentsname
  \renewcommand*\contentsname{Table of contents}
\else
  \newcommand\contentsname{Table of contents}
\fi
\ifdefined\listfigurename
  \renewcommand*\listfigurename{List of Figures}
\else
  \newcommand\listfigurename{List of Figures}
\fi
\ifdefined\listtablename
  \renewcommand*\listtablename{List of Tables}
\else
  \newcommand\listtablename{List of Tables}
\fi
\ifdefined\figurename
  \renewcommand*\figurename{Figure}
\else
  \newcommand\figurename{Figure}
\fi
\ifdefined\tablename
  \renewcommand*\tablename{Table}
\else
  \newcommand\tablename{Table}
\fi
}
\@ifpackageloaded{float}{}{\usepackage{float}}
\floatstyle{ruled}
\@ifundefined{c@chapter}{\newfloat{codelisting}{h}{lop}}{\newfloat{codelisting}{h}{lop}[chapter]}
\floatname{codelisting}{Listing}
\newcommand*\listoflistings{\listof{codelisting}{List of Listings}}
\makeatother
\makeatletter
\makeatother
\makeatletter
\@ifpackageloaded{caption}{}{\usepackage{caption}}
\@ifpackageloaded{subcaption}{}{\usepackage{subcaption}}
\makeatother
\usepackage{bookmark}
\IfFileExists{xurl.sty}{\usepackage{xurl}}{} % add URL line breaks if available
\urlstyle{same}
\hypersetup{
  pdftitle={Bioinformatic Class Spring 2026 Week2},
  pdfauthor={JM},
  colorlinks=true,
  linkcolor={blue},
  filecolor={Maroon},
  citecolor={Blue},
  urlcolor={Blue},
  pdfcreator={LaTeX via pandoc}}


\title{Bioinformatic Class Spring 2026 Week2}
\author{JM}
\date{}
\begin{document}
\maketitle


\section{Introduction to RStudio and
Rmarkdown}\label{introduction-to-rstudio-and-rmarkdown}

This is an (R)markdown document. Markdown is a plain-text format that
allows both humans and machines to read it easily. Rmarkdown is an
extension that allows for an interactive coding environment and easy
generation of reports while maintaining the simplicity and flexibility
of markdown. For more details on using Rmarkdown see
\url{http://rmarkdown.rstudio.com}.

\subsection{RStudio}\label{rstudio}

\subsubsection{.RData}\label{rdata}

Go to Tools \textgreater{} Global Options. In the window that pops up
under \textbf{Workspace}, uncheck ``Restore .RData into workspace at
startup'' and set ``Save workspace to .RData on exit:'' to
\textbf{Never}

\subsubsection{Tour}\label{tour}

\begin{itemize}
\tightlist
\item
  Source
\item
  Console
\item
  Environment/History
\item
  Files/Plots/Packages/Help
\end{itemize}

\subsection{Rmarkdown}\label{rmarkdown}

\subsubsection{How it Works}\label{how-it-works}

R Markdown documents consist of mixtures of two main sections/types:
simple plain text and code chunks. What you're reading right now is in
plain text. Here you can type whatever you want and add some simple
formatting (more on that later). Below, the gray boxes are code chunks.
With RStudio as an interface you can run the code directly either by:

\begin{enumerate}
\def\labelenumi{\arabic{enumi}.}
\tightlist
\item
  Clicking the green arrow on the upper right hand side of the code
  chunk
\item
  Clicking ``Run Current Chunk'' in the ``Run'' drop down menu above
\item
  Hitting \texttt{Command} + \texttt{Shift} + \texttt{Return} on a Mac
  or \texttt{Ctrl} + \texttt{Shift} + \texttt{Enter} on a PC
\end{enumerate}

\textbf{Your cursor must be between the backticks for RStudio to run the
correct chunk}

\begin{Shaded}
\begin{Highlighting}[]
\FunctionTok{head}\NormalTok{(iris)}
\end{Highlighting}
\end{Shaded}

\begin{verbatim}
  Sepal.Length Sepal.Width Petal.Length Petal.Width Species
1          5.1         3.5          1.4         0.2  setosa
2          4.9         3.0          1.4         0.2  setosa
3          4.7         3.2          1.3         0.2  setosa
4          4.6         3.1          1.5         0.2  setosa
5          5.0         3.6          1.4         0.2  setosa
6          5.4         3.9          1.7         0.4  setosa
\end{verbatim}

\begin{Shaded}
\begin{Highlighting}[]
\FunctionTok{plot}\NormalTok{(iris}\SpecialCharTok{$}\NormalTok{Sepal.Length, iris}\SpecialCharTok{$}\NormalTok{Sepal.Width)}
\end{Highlighting}
\end{Shaded}

\pandocbounded{\includegraphics[keepaspectratio]{BioinfoClass_2026_week2_rmarkdown_files/figure-pdf/unnamed-chunk-1-1.pdf}}

\subsubsection{How to Make a Code Chunk}\label{how-to-make-a-code-chunk}

To make a code chunk, type three backticks (`), follow by a lowercase r
between curly braces (\{\}), hit \texttt{Enter}, and finish with three
more backticks on the next line.

\begin{Shaded}
\begin{Highlighting}[]
\CommentTok{\# this is a code chunk!}
\end{Highlighting}
\end{Shaded}

Inside a code chunk you can use a hashtag/ \texttt{\#} to leave comments
that won't be executed. This is useful to leave notes for yourself
(about decisions you made in your analysis or what the output of code
is) and also to stop part of your code from executing when you're
debugging it.

\begin{Shaded}
\begin{Highlighting}[]
\CommentTok{\# this plot will be executed}
\FunctionTok{plot}\NormalTok{(iris}\SpecialCharTok{$}\NormalTok{Sepal.Length, iris}\SpecialCharTok{$}\NormalTok{Sepal.Width)}
\end{Highlighting}
\end{Shaded}

\pandocbounded{\includegraphics[keepaspectratio]{BioinfoClass_2026_week2_rmarkdown_files/figure-pdf/unnamed-chunk-3-1.pdf}}

\begin{Shaded}
\begin{Highlighting}[]
\CommentTok{\# this one won\textquotesingle{}t}
\CommentTok{\# plot(iris$Petal.Length, iris$Petal.Width)}
\end{Highlighting}
\end{Shaded}

\subsection{How to Format Your Text}\label{how-to-format-your-text}

Markdown allows the use of simple symbols to format your plain text.

\subsubsection{Format Text Appearance}\label{format-text-appearance}

Throughout this document, there are a bunch of lines with \#s in front
of them. The \# sign creates headings, lines of the document that are
emphasized through some combination of, depending on the interpreter,
larger font size, larger line weight/bolding, italicizing, and
underling. One \# creates the largest header, and with every \#\# added,
headers become smaller and less emphasized.

\section{Heading 1}\label{heading-1}

\subsection{Heading 2}\label{heading-2}

\subsubsection{Heading 3}\label{heading-3}

\paragraph{Heading 4}\label{heading-4}

\subparagraph{Heading 5}\label{heading-5}

Heading 6

\begin{center}\rule{0.5\linewidth}{0.5pt}\end{center}

To make text italicized add one * on either side of it.

\textbf{This text will render italicized.}

\begin{center}\rule{0.5\linewidth}{0.5pt}\end{center}

To make text bold, add two ** to either side of it.

\textbf{This text will render bold}

\subsubsection{Lists}\label{lists}

You can either make an ordered list with numbers, ex:

\begin{enumerate}
\def\labelenumi{\arabic{enumi}.}
\tightlist
\item
  Item 1
\item
  Item 2
\item
  Item 3
\item
  Item 4
\end{enumerate}

Or make an unordered list, a bullet-point list, ex:

\begin{itemize}
\tightlist
\item
  Item 1
\item
  Item 2
\item
  Item 3
\end{itemize}

\subsubsection{Miscellaneous Formatting}\label{miscellaneous-formatting}

Adding three dashes (-) creates a horizontal line break, ex:

\begin{center}\rule{0.5\linewidth}{0.5pt}\end{center}

You can add inline code with single backticks (`) around it, ex:
\texttt{head(iris)}. This code won't execute, but will look like a small
code chunk inside the text to an interpreter.

\begin{center}\rule{0.5\linewidth}{0.5pt}\end{center}

You can make a link with square brackets {[}{]} followed by parentheses
(), with the link text in the brackets and the link address in the
parentheses.

\href{https://github.com/Wistar-bioinfo/BioinfoClass_Spring_2026}{All
the material for the class is here}

\begin{center}\rule{0.5\linewidth}{0.5pt}\end{center}

If you want to link to a picture, you use an exclamation point in front
of the square brackes, like
\includegraphics[width=5.33333in,height=\textheight,keepaspectratio]{./blue_box_logo-Converted_1024x366.png}

\subsection{Output R Markdown in Another File
Format}\label{output-r-markdown-in-another-file-format}

You can output R Markdown files as html, pdf, or word files by clicking
\texttt{Knit} above. By default, R Markdown will give you an html file,
but you can select the format in the drop down menu beside the
\texttt{Knit} button. Unlike using an interpreter to make the document
look prettier, knitting an R Markdown document will run all the code in
code chunks and show you the code and the results of the code in the
final document.




\end{document}
